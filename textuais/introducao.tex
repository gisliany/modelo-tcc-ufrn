\chapter[Introdução]{Introdução}
\label{ch:introdução}
Escreva aqui seu capítulo introdutório. 
Ele pode conter figuras, tabelas e subseções. Exemplo de uma citação indireta \cite{yu2011new}. De acordo com~\citeonline{ali2013smart}, isso é uma citação indireta com menção do autor incluída na frase. Além disso, toda imagem, como a \autoref{fig:rede_convencional}, deve ser mencionada e explicada no texto. Imagens do autor, tem na fonte o texto "Elaborado pelo autor". Além disso, as legendas das imagens devem ser autoexplicativas, de modo que façam sentido sozinhas na lista de figuras.

\begin{figure}[h!]
	\includegraphics[width=0.9\textwidth, keepaspectratio=true]{rede_convencional}
	\centering
	\caption[Esquemático de uma rede elétrica convencional.]{Esquemático de uma rede elétrica convencional.}
	\fonte{\citeonline[p. 15]{cgee}.}
	\label{fig:rede_convencional}
\end{figure}
\FloatBarrier

A seguir, exemplo de citação direta com menos de três linhas. Como "o mercado de energia elétrica está baseado em tarifas fixas e limitações de informações em tempo real sobre gerenciamento da rede e da carga" \cite[p. 15]{cgee}, o consumidor acaba, então, não tendo como optar por fornecimentos elétricos mais adequados. 


\section{Uma subseção explicativa}

Lorem ipsum, uma citação direta 

\begin{citacao}[brazil]
[...] redes elétricas que podem, de forma inteligente, integrar o comportamento e as ações de todos os usuários conectados a ela, como geradores, consumidores e os que desempenham as duas funções, para entregar, eficientemente, um fornecimento de eletricidade sustentável, econômico e seguro \cite[p. 51, tradução livre]{yu2011new}.
\end{citacao}

Para compreender melhor as grandes mudanças e os benefícios gerados pelas \textit{Smart Grids} no contexto do fornecimento elétrico, a \autoref{tab-comparativa} traz um breve comparativo entre as redes tradicionais e as redes inteligentes.

\begin{table}[!ht]
\centering
% com o comando resizebox é possível definir a largura da tabela
%\resizebox{\textwidth}{!}{%
\begin{tabular}{lc}
\toprule
\multicolumn{1}{c}{\textbf{Variável}} & \multicolumn{1}{c}{\textbf{Valor}} \\
\midrule
Variável 01 & 0,7 \\
Variável 02 & 0,1 \\
Variável 03 & 1,0 \\
Variável 04 & 0,5 \\
Variável 05 & 0,9 \\
Variável 06 & 2,0 \\
\bottomrule
\end{tabular}%
%}
\caption{Comparação entre redes elétricas convencionais e redes elétricas inteligentes}
\label{tab-comparativa}
\fonte{\citeonline[p. 28, tradução nossa]{ali2013smart}}
\end{table}

\section{Trabalhos Relacionados}
Aqui são mencionados os trabalhos já existentes na literatura que descrevem como se encontra atualmente o ramo científico que você está pesquisando. Existem trabalhos similares ao seu? Mencione-os, explicando o que os autores fizeram e como fizeram. Ao final, é interessante evidenciar o que o seu trabalho traz como novidade e contribuição.
\lipsum[1-1]

\section{Motivação}
O que lhe motiva a realizar este trabalho.

\section{Objetivos}
Objetivo geral e específicos. Objetivos devem ser escritos com verbo no infinivo. Em linhas gerais, os objetivos específicos são "os passos menores" para se alcançar o objetivo geral.

\section{Estrutura do Trabalho}
Este trabalho apresenta uma introdução sobre o tema, mostrando os fatores que motivam a implantação da ideia, além da justificativa e dos objetivos. Em sequência, o \autoref{ch:cap2} aborda (...). O \autoref{ch:cap3}, por sua vez, explica a metodologia para ..., enquanto o \autoref{ch:cap4} trata de (...). O \autoref{ch:cap5} apresenta (...). Por fim, o \autoref{ch:cap6} traz as principais conclusões e contribuições deste trabalho.