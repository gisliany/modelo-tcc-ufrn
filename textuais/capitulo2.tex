\chapter[Capítulo 2]{Capítulo 2}
\label{ch:cap2}

Neste capítulo, são apresentados os conhecimentos necessários para (...).

\section[\textit{Seção}]{\textit{Seção}}
\lipsum[1-2]

Itens em latex:
\begin{itemize}
	\item texto 1;
	\item texto 2;
	\item texto 3;
\end{itemize}

\lipsum[2-2]

A seguir, o Algoritmo~\ref{algoritmo1} mostra o procedimento descrito de seleção de variáveis sob a forma de pseudocódigo. 

\begin{algorithm}[ht]
\SetAlgoLined
\Dados{$dataset$}
\Saida{Conjunto das melhores variáveis regressoras para o regressando y}
    \BlankLine
    $k \leftarrow 1$\;
    $limite\_fiv \leftarrow 5$\;
    \Enqto{k $\leq$ dataset.numero\_regressores}{
        \tcp{seleção das k melhores variáveis usando Teste F}
        $k\_regressores \leftarrow$ \textsl{selecionarKMelhores(dataset.x, dataset.y, ``testeF'', k)}\;
        
        \BlankLine
        \tcp{análise de multicolinearidade}
        \Para{\emph{cada} regressor \emph{em} k\_regressores}{
            \Se{\textsl{calcularFIV(regressor)} $\leq$ limite\_fiv}{
                $regressores\_selecionados$.\textsl{adicionar(regressor)}\;
            }
        }
        
        \BlankLine
        \tcp{modelo de regressão (ajuste pelo MQO) e coeficiente R\textsuperscript{2}}
        $modelo \leftarrow$\textsl{MQO(dataset.y, regressores\_selecionados)}\;
        $resultados.$\textsl{adicionar(k, regressores\_selecionados, modelo.r2)}\;
        
    $k \leftarrow k + 1$;
 }
 \Retorna{resultados.\textsl{obterRegressoresComR2Maximo()}\;}
 \caption{Seleção de variáveis para análise de regressão}
 \label{algoritmo1}
\end{algorithm}

\subsection{Subseção}

\lipsum[2-4]