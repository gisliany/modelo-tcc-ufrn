\documentclass[
	12pt,				% tamanho da fonte		
	oneside,
	a4paper,			% tamanho do papel.
	chapter=TITLE,
	%sumario=tradicional,
	english,			% idioma adicional
	brazil,				% idioma principal do documento
	]{abntex2}

% ---
% PACOTES
% ---
\usepackage{lmodern}			% Usa a fonte Latin Modern
\usepackage[T1]{fontenc}		% Selecao de codigos de fonte.
\usepackage[utf8]{inputenc}		% Codificacao do documento (conversão automática dos acentos)
\usepackage{indentfirst}		% Indenta o primeiro parágrafo de cada seção.
\usepackage{color}				% Controle das cores
\usepackage{graphicx}			% Inclusão de gráficos
\graphicspath{{figuras/},{figuras/Intro/},{figuras/CAP2/},{figuras/CAP3/},{figuras/CAP4/},{figuras/CAP5/}}
\usepackage{microtype} 			% para melhorias de justificação
\usepackage{booktabs}
\usepackage{float} 				% Set posição da figura
\usepackage[bottom]{footmisc} 
\usepackage{subfig} 			% Inserir subfiguras
\usepackage[table,xcdraw]{xcolor} 	% Cor de preenchimento das tabelas
\usepackage{multirow} 			%mesclar cel em tabelas
\usepackage{verbatim}			%Inserir codigos fontes e comentários em massa
\usepackage[brazilian,hyperpageref]{backref}
\usepackage[alf]{lib/abntex2cite}
\usepackage{lipsum}				% para geração de dummy text
\usepackage{amsmath}
\usepackage[bottom]{footmisc}
\usepackage{footnote}
%\usepackage{fnpos}
%\usepackage{ftnxtra}
\usepackage{listings} 			%Inserir códigos fontes
\usepackage{rotating} 			%Rotação de páginas
\usepackage{placeins}			% Forçar o posicionamento da figura
\usepackage[top=3cm, bottom=2cm, left=3cm, right=2cm]{geometry} % Margens


% ---
% Informações de dados para CAPA e FOLHA DE ROSTO
% ---

\titulo{{\normalfont \textbf{Título do seu Trabalho}}}
\autor{Fulano de Tal}
\local{Natal -- RN}
\data{Dezembro de 2017}
\instituicao{
  Universidade Federal do Rio Grande do Norte -- UFRN
  \par
  Departamento de Engenharia de Computação e Automação -- DCA
  \par
  Curso de Engenharia de Computação

}
\tipotrabalho{Relatório técnico}

\preambulo{Trabalho de Conclusão de Curso de Engenharia de Computação da Universidade Federal do Rio Grande do Norte, apresentado como requisito parcial para a obtenção do grau de Bacharel em Engenharia de Computação
\newline 
\newline 
Orientador: John Doe}

% Configurações de aparência do PDF final
% alterando o aspecto da cor azul
\definecolor{blue}{RGB}{41,5,195}
% informações do PDF
\makeatletter
\hypersetup{
     	%pagebackref=true,
		pdftitle={\@title}, 
		pdfauthor={\@author},
    	pdfsubject={\imprimirpreambulo},
	    pdfcreator={LaTeX with abnTeX2},
		pdfkeywords={abnt}{latex}{abntex}{abntex2}{relatório técnico}, 
		colorlinks=true,       		% false: boxed links; true: colored links
    	linkcolor=black,          	% color of internal links
    	citecolor=black,        		% color of links to bibliography
    	filecolor=magenta,      		% color of file links
		urlcolor=blue,
		bookmarksdepth=4
}
\makeatother
% Espaçamentos entre linhas e parágrafos 
\setlength{\parindent}{1.25cm} % Tamanho do parágrafo
\setlength{\parskip}{0.2cm}	% Controle do espaçamento entre um parágrafo e outro

%\onelineskip % Controle do espaçamento entre um parágrafo e outro

\makeindex % compila o indice

% Início do documento
% ----
\begin{document}

% Seleciona o idioma do documento (conforme pacotes do babel)
%\selectlanguage{english}
\selectlanguage{brazil}

% Retira espaço extra obsoleto entre as frases.
\frenchspacing 

% ----------------------------------------------------------
% ELEMENTOS PRÉ-TEXTUAIS
% ----------------------------------------------------------
\pretextual

% Capa
\imprimircapa

% Folha de rosto
\imprimirfolhaderosto

% Inserir folha de aprovação
\begin{folhadeaprovacao}
	
	\begin{center}
		{\ABNTEXchapterfont\large\imprimirautor}
		
		\vspace*{\fill}\vspace*{\fill}
		\begin{center}
			\ABNTEXchapterfont\bfseries\Large\imprimirtitulo
		\end{center}
		\vspace*{\fill}
		
		\hspace{.45\textwidth}
		\begin{minipage}{.5\textwidth}
			\imprimirpreambulo
		\end{minipage}%
		\vspace*{\fill}
	\end{center}
	
	Trabalho aprovado. \imprimirlocal, 08 de Dezembro de 2017:
	
	
	\setlength{\ABNTEXsignwidth}{14cm}
	\assinatura{\textbf{Prof. Dr. John Doe - Orientador} \\ UFRN} 
	\assinatura{\textbf{Prof. Dr. Cicrano da Silva - Coorientador} \\ UFRN}
	\assinatura{\textbf{MSc. Alguém externo - Convidado} \\ Empresa ou instituição }
	
	\begin{center}
		\vspace*{0.5cm}
		{\large\imprimirlocal}
		\par
		{\large\imprimirdata}
		\vspace*{1cm}
	\end{center}
	
\end{folhadeaprovacao}

% Dedicatória
\begin{dedicatoria}
	\vspace*{\fill}
	\centering
	\noindent
	\textit{Escreva aqui sua dedicatória}
	 \vspace*{\fill}
\end{dedicatoria}

% Agradecimentos
\begin{agradecimentos}
Escreva aqui seus agradecimentos.

\end{agradecimentos}

% ---
% Epígrafe
% ---
\begin{epigrafe}
	\vspace*{\fill}
	\begin{flushright}
		\textit{``Feliz o homem que encontrou a sabedoria e alcançou o entendimento,\\
			porque a sabedoria vale mais do que a prata, \\
			e dá mais lucro que o ouro."\\
			(Bíblia Sagrada, Provérbios 3, 13-14)}
	\end{flushright}
\end{epigrafe}

% RESUMO
% resumo na língua vernácula (obrigatório)
\setlength{\absparsep}{18pt} % ajusta o espaçamento dos parágrafos do resumo
\begin{resumo}

Escreva seu resumo aqui. Ele deve ser parágrafo único e sem récuo na primeira linha. O resumo deve tratar das informações gerais do trabalho, trazendo tema, objetivo(s), metodologia, principais resultados e considerações finais. Em geral, no resumo não cabem citações estruturas enumeradas são incomuns.
 
 
 \noindent
 \textbf{Palavras-chaves}: palavra1. palavra2. palavra3. 
\end{resumo}
% ---
% resumo em inglês
\begin{resumo}[Abstract]
	\begin{otherlanguage*}{english}
	
	Write here your abstract considering the same rules.
	\lipsum[1-1]
	
	\vspace{\onelineskip}
	\noindent 
	\textbf{Keywords}: keyword1. keyword2. keyword3.
	\end{otherlanguage*}
\end{resumo}

% ---
% inserir lista de ilustrações
\pdfbookmark[0]{\listfigurename}{lof}
\listoffigures*
\cleardoublepage

% inserir lista de tabelas
\pdfbookmark[0]{\listtablename}{lot}
\listoftables*
\cleardoublepage

% inserir lista de abreviaturas e siglas
\begin{siglas}
\item[HTML] \textit{HyperText Markup Language}
\item[JSON] \textit{JavaScript Object Notation}
\item[REST] \textit{Representational State Transfer}
\end{siglas}

% inserir lista de símbolos
\begin{simbolos}
  \item[$ \Gamma $] Letra grega Gama
  \item[$ \Lambda $] Lambda
  \item[$ \zeta $] Letra grega minúscula zeta
  \item[$ \in $] Pertence
\end{simbolos}

% inserir o sumario
\pdfbookmark[0]{\contentsname}{toc}
\tableofcontents*
\cleardoublepage

\textual

% Capitulo 1: Introdução
\chapter[Introdução]{Introdução}
\label{ch:introdução}
Escreva aqui seu capítulo introdutório. Ele pode conter figuras, tabelas e subseções. Exemplo de uma citação indireta \cite{yu2011new}, e da \autoref{fig:rede_convencional}. Imagens do autor, tem na fonte o texto "Elaborado pelo autor".

\begin{figure}[h!]
	\includegraphics[width=0.9\textwidth, keepaspectratio=true]{rede_convencional}
	\centering
	\caption[Esquemático de uma rede elétrica convencional.]{Esquemático de uma rede elétrica convencional.}
	\fonte{\cite[p. 15]{cgee}.}
	\label{fig:rede_convencional}
\end{figure}
\FloatBarrier

Exemplo de citação direta com menos de três linhas. Como "o mercado de energia elétrica está baseado em tarifas fixas e limitações de informações em tempo real sobre gerenciamento da rede e da carga" \cite[p. 15]{cgee}, o consumidor acaba, então, não tendo como optar por fornecimentos elétricos mais adequados. 


\section{Uma subseção explicativa}

Lorem ipsum, uma citação direta 

\begin{citacao}[brazil]
[...] redes elétricas que podem, de forma inteligente, integrar o comportamento e as ações de todos os usuários conectados a ela, como geradores, consumidores e os que desempenham as duas funções, para entregar, eficientemente, um fornecimento de eletricidade sustentável, econômico e seguro \cite[p. 51, tradução livre]{yu2011new}.
\end{citacao}

Para compreender melhor as grandes mudanças e os benefícios gerados pelas \textit{Smart Grids} no contexto do fornecimento elétrico, a \autoref{tab-comparativa} traz um breve comparativo entre as redes tradicionais e as redes inteligentes.

\begin{table}[!ht]
\centering
\resizebox{\textwidth}{!}{%
\begin{tabular}{ll}
\hline
\multicolumn{1}{c}{\textbf{Redes Elétricas Tradicionais}} & \multicolumn{1}{c}{\textbf{Redes Elétricas Inteligentes}}                 \\ \hline
\rowcolor[HTML]{DDDDDD} 
Eletromecânica, estado sólido                             & Digital/Microprocessadores                                                \\
Unidirecional e localmente bidirecional                   & Global/comunicação bidirecional integrada                                 \\
\rowcolor[HTML]{DDDDDD} 
Geração centralizada                                      & Acomoda geração distribuída                                               \\
{Controle, monitoramento e proteção limitados}  & WAMPAC, proteção adaptativa \\
\rowcolor[HTML]{DDDDDD} 
"Cega"                                                    & Auto-monitoramento                                                        \\
Recuperação manual                                        & Auto-reconfigurável                                                       \\
\rowcolor[HTML]{DDDDDD} 
Checagem manual de equipamentos                           & Monitoração remota de equipamentos                                        \\
Sistema de controle de contingências limitado             & Sistema de controle pervasivo                                             \\
\rowcolor[HTML]{DDDDDD} 
Confiabilidade estimada                                   & Confiabilidade preditiva                                                 
\end{tabular}%
}
\caption{Comparação entre redes elétricas convencionais e redes elétricas inteligentes}
\label{tab-comparativa}
\fonte{\cite[p. 28, tradução nossa]{ali2013smart}}
\end{table}

\section{Trabalhos Relacionados}
\lipsum[1-1]

\section{Motivação}
O que lhe motiva a realizar este trabalho.

\section{Objetivos}
Objetivo geral e específicos.

\section{Estrutura do Trabalho}
Este trabalho apresenta uma introdução sobre o tema, mostrando os fatores que motivam a implantação da ideia, além da justificativa e dos objetivos. Em sequência, o \autoref{ch:cap2} aborda (...). O \autoref{ch:cap3}, por sua vez, explica a metodologia para ..., enquanto o \autoref{ch:cap4} trata de (...). O \autoref{ch:cap5} apresenta (...). Por fim, o \autoref{ch:cap6} traz as principais conclusões e contribuições deste trabalho.

% Capitulo 2
\chapter[Capítulo 2]{Capítulo 2}
\label{ch:cap2}

Neste capítulo, são apresentados os conhecimentos necessários para (...).

\section[\textit{Seção}]{\textit{Seção}}
\lipsum[1-2]

Itens em latex:
\begin{itemize}
	\item texto 1;
	\item texto 2;
	\item texto 3;
\end{itemize}

\lipsum[2-2]

A seguir, o Algoritmo~\ref{algoritmo1} mostra o procedimento descrito de seleção de variáveis sob a forma de pseudocódigo. 

\begin{algorithm}[ht]
\SetAlgoLined
\Dados{$dataset$}
\Saida{Conjunto das melhores variáveis regressoras para o regressando y}
    \BlankLine
    $k \leftarrow 1$\;
    $limite\_fiv \leftarrow 5$\;
    \Enqto{k $\leq$ dataset.numero\_regressores}{
        \tcp{seleção das k melhores variáveis usando Teste F}
        $k\_regressores \leftarrow$ \textsl{selecionarKMelhores(dataset.x, dataset.y, ``testeF'', k)}\;
        
        \BlankLine
        \tcp{análise de multicolinearidade}
        \Para{\emph{cada} regressor \emph{em} k\_regressores}{
            \Se{\textsl{calcularFIV(regressor)} $\leq$ limite\_fiv}{
                $regressores\_selecionados$.\textsl{adicionar(regressor)}\;
            }
        }
        
        \BlankLine
        \tcp{modelo de regressão (ajuste pelo MQO) e coeficiente R\textsuperscript{2}}
        $modelo \leftarrow$\textsl{MQO(dataset.y, regressores\_selecionados)}\;
        $resultados.$\textsl{adicionar(k, regressores\_selecionados, modelo.r2)}\;
        
    $k \leftarrow k + 1$;
 }
 \Retorna{resultados.\textsl{obterRegressoresComR2Maximo()}\;}
 \caption{Seleção de variáveis para análise de regressão}
 \label{algoritmo1}
\end{algorithm}

\subsection{Subseção}

\lipsum[2-4]

% Capitulo 3
\chapter[Capítulo 3]{Capítulo 3}
\label{ch:cap3}

Fórmulas e itens:

\begin{itemize}
    \item Primeiro
    \subitem - $C_1 > F_1 + F_2 $
    \item Segundo
    \subitem - $C_2 > F_1 + F_2 - C_1$
\end{itemize}

\begin{equation} \label{eq1}
R(t) = P(T > t) = 1 - F(t)
\end{equation}

\section{Seção 1} \label{secao1}

Use \textit{labels} para criar links para as seções, como \autoref{secao1}.

\lipsum[5-8]

% Capitulo 4
\chapter[Capítulo 4]{Capítulo 4}
\label{ch:cap4}
\lipsum[3-5]

\section{Seção}

\lipsum[1-2]

\subsection{Subseção}
\lipsum[3-5]

\section{Seção 2}\label{secao2}

O SHAPE é a sigla em inglês para \textit{Symbolic Hierarchical Automated Reliability and Performance Evaluator}. Veja a  \autoref{fig:sharpe}.

\begin{figure}[!h]
	\includegraphics[width=1.0\textwidth, keepaspectratio=true]{sharpe}
	\centering
	\caption[\textit{Print screen} do SHARPE em linha de comando e em interface gráfica.]{\textit{Print screen} do SHARPE em linha de comando e em interface gráfica.}
	\fonte{Elaborada pela autora.}
	\label{fig:sharpe}
\end{figure}

% Capitulo 5
\chapter[Capítulo 5]{Capítulo 5}
\label{ch:cap5}

\lipsum[2-3]

\section{Seção}

\lipsum[3-4]

% Conclusão
\chapter[Conclusão]{Conclusão}
\label{ch:cap6}

Escreva suas conclusões, limitações do seu trabalho, contribuições, trabalhos futuros, etc.



% ----------------------------------------------------------
% ELEMENTOS PÓS-TEXTUAIS

% 
\postextual

% Referências bibliográficas
%\addcontentsline{toc}{chapter}{Referências Bibliográficas}

\bibliographystyle{abntex2-alf}
%\bibliographystyle{unsrt}
\bibliography{bibliografia/referencias}

\end{document}
